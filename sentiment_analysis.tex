\documentclass[12pt,a4paper,titlepage]{article}

% Napraviti da se normalno može koristiti hrvatski jezik
\usepackage[croatian]{babel}
\usepackage[utf8]{inputenc}
\usepackage[T1]{fontenc}


\usepackage{hyperref} % Linkanje referenci (npr. bibliografija)
\usepackage{float}    % Omogućiti inline slike u tekstu
\usepackage[nottoc,numbib]{tocbibind} % Numeriranje bibliografije

\usepackage{color}    % Omogućiti boje
\usepackage{multicol} % Omogućiti više kolumna teksta (npr. 2)
\usepackage{comment}
\usepackage{amsmath}
\usepackage{amsfonts}
\usepackage{amssymb}
\usepackage{graphicx}
\usepackage{parskip}

\title{Analiza stava\\(Projektni prijedlog)}
\author{Lana Arzon\\Janko Marohnić\\Antonio Kovačić}
\date{Zagreb, 2014.}

\DeclareMathOperator*{\argmax}{arg\,max}

\begin{document}

\maketitle

\tableofcontents

\newpage

\section{Uvodni opis problema}

\textbf{Analizu stava} (sentiment analysis, opinion minning) možemo opisati kao skup statističkih metoda i metoda strojnog učenja, odnosno obrade prirodnog jezika, kojima se izdvaja, identificira, odnosno zaključuje o stavu izraženom u određenoj tekstualnoj jedinici (rečenici, odnosno odlomku).

Primjene analize stava mogu biti razne:

\begin{enumerate}
  \item Otkrivanje stava o političkim opcijama (vladi, stranci)
  \item Otkrivanje stava u marketingu -- mišljenje o određenom proizvodu
  \item Otkrivanje stava o određenom glazbenom sastavu ili određenoj skladbi (glazbenom uratku)
  \item Otkrivanje stava o određenom filmu
\end{enumerate}

Mi smo odabrali otkrivanje stava o filmovima (4. točka). Primjena takvog našega problema bi mogla biti iduća: klasifikacijom što određeni korisnik voli gledati (koji žanr) i na temelju onoga što je korisnik rekao o filmovima sličnog sadržaja -- možemo zaključiti hoće li se određenom korisniku $A$ svidjeti određeni film ili ne (i s obzirom na to -- ponuditi mu), osim onoga što je rekao korisnik $A$, možemo vidjeti i stav o filmu ostalih korisnika koji su već pogledali taj film, a koji su davali sličan stav stavu korisnika $A$ o drugim filmovima koje je pogledao korisnik $A$ te na temelju toga zaključiti bi li se korisniku $A$ ponuda svidjela.

\section{Cilj i hipoteze istraživanja problema}

Naš zadatak se bazira na slijedećem: otkriti da li je stav izrečen u nekom tekstu pozitivan ili negativan; konkretno, je li filmska kritika pozitivna ili negativna. Cilj našega projekta je što bolja klasifikacija stava. Orijentirat ćemo se tako da ispitujemo koja je metoda bolja za klasifikaciju teksta u toj aplikacijskoj domeni.
% ovo staviti u metodologiju i slično...

\section{Pregled dosadašnjih istraživanja}

U pregledu dosadašnjih istraživanja najviše su se koristile metode zvane "Naivni Bayes", "Metoda potpornih vektora" (\textsc{SVM}) i "Maksimalna entropija". Prema \cite{Pang:2002:TUS:1118693.1118704} u poglavlju 6. (\textit{Figure 3}) jasno se vidi da \textsc{SVM} u većini slučajeva daje najbolje rezultate. Također, u \cite{stan} je pokazano da se za različite varijacije \textsc{SVM} metode mogu postići različiti (bolji) rezultati. Osim obične metode \textit{Bag of Words} opisane u istraživanju \cite{stan}, koristile su se takozvane \textsc{LDA} i \textsc{TF-IDF} metode - za\textsc{TF-IDF} su globalno gledano dobiveni neznatno bolji statistički rezultati, dok za \textsc{LDA} lošiji (\textit{Figure 3.} i \textit{Figure 4}). Također u \cite{SaLAD:LAS}, uvedena su neka poboljšanja na metodu naivnog Bayesa (Bernoullijev naivni Bayes), poboljšani su rezultati profinjavanjem negacija, manipulacijama sa $n$-gramima te samim izborom značajki. U istraživanju \cite{maas-EtAl:2011:ACL-HLT2011} se spominje još metoda \textsc{LSA} za odabir značajki. Prigodom odabira značajki, mogu se vidjeti razne varijacije -- osim pojavnosti same riječi u kritikama, može se promatrati i njena pozicija u rečenici ili čak kontekst što je prikazano u \cite{Pang:2002:TUS:1118693.1118704}.

\section{Materijali, metodologija i plan istraživanja}

Kao izvor podataka korisit ćemo \cite{dataset}. Podatke ćemo možda morati prilagoditi (odnosno prilagoditi format u kojem je tekst zapisan). Ovaj skup se sastoji od dokumenata (kritike filmova) koji su podijeljeni na dvije klase -- pozitivne i negativne kritike. Pomoću njega želimo naučiti naš klasifikator da za dani nepoznati dokument odredi da li je riječ o pozitivnoj ili negativnoj kritici -- nadzirano učenje.

Kao značajke ćemo koristiti riječi. Služit ćemo se jednostavnom metodom \textit{Bag of words} koja se temelji samo na broju pojavljivanja značajke, ne uzimajući u obzir poredak i ostale značajke. Dakle, pretpostavljamo da su značajke nezavisne.

Projekt ćemo realizirati u programskom jeziku Python korištenjem alata NLTK (Natural Language Toolkit). 

Kao bazni postupak za praktično provođenje našeg projekta izabrali smo:

\begin{enumerate}
  \item Razbijanje tekstualne cijeline na tokene (riječi)
  \item Izlučivanje entiteta -- bitnih značajki -- iz rečenica
    \begin{itemize}
      \item To uključuje identificiranje relevantnih imenica, glagola, priloga i sl, a zanemrivanje onih značajki koje se često pojavljuju, a nisu bitne za značenje same kritike (npr. veznici).
    \end{itemize}
  \item Izbacivanje objektivnih dijelova rečenica
    \begin{itemize}
      \item Na primjer, iako rečenica "Taj dan John je naišao na prelijepu djevojku." sadrži riječ "lijep", ne govori nam ništa o autorovom mišljenju, i ta rečenica može čak biti dio negativne kritike.
    \end{itemize}
  \item Klasifikacija korištenjem različitih klasifikatora
    \begin{itemize}
      \item Naivni Bayes
      \item Metoda potpornih vektora
      \item Metoda maksimalne entropije
    \end{itemize}
\end{enumerate}

\subsection{Naivni Bayes}

Neka imamo dokument $d$ i klasu dokumenta $c$ (u našem slučaju je $c \in \{\text{kritika je negativna}, \text{kritika je pozitivna}\}$). Tada $P(c|d)$ predstavlja "vjerojatnost da je dokument $d$ klase $c$". Po Bayesovoj formuli, imamo

\[
  P(c|d) = \frac{P(d|c)P(c)}{P(d)}.
\]

Neka je $f_1, f_2, \ldots, f_n$ skup značajki koji se mogu pojaviti u dokumentu, i neka je $n_i(d)$ broj pojave $f_i$ u dokumentu $d$. Tada imamo

\[
  P_\text{NB}(c|d) := \frac{P(c)\left(\prod_{i=1}^{m} P(f_i|c)^{n_i(d)}\right)}{P(d)}.
\]

\subsection{Maksimalna entropija}

Maksimalna entropija ponekad, ali ne uvijek, daje bolje rezultate od Naivnog Bayesa. Njena procjena $P(c|d)$ ima sljedeću eksponencijalnu formu:

\[
  P_\text{NB}(c|d) := \frac{1}{Z(d)} \exp \left(\sum_i \lambda_{i,c} F_{i,c}(d,c)\right),
\]

gdje je

\begin{itemize}
  \item $Z(d)$ normalizirajuća funkcija
  \item $F_{i,c}$ karakteristična funkcija za značajku $f_i$ i klasu $c$:
    \[
      F_{i,c}(d,c') := \begin{cases}
                         1, & n_i(d) > 0 \text{ i } c' = c\\
                         0, & \text{inače}
                       \end{cases}
    \]
  \item $\lambda_{i,c}$ su težine značajki (veliki $\lambda_{i,c}$ znači da se $f_i$ smatra jakim indikatorom za klasu $c$)
\end{itemize}

\subsection{Metoda potpornih vektora}

Metoda potpornih vektora pokazala se vrlo učinkovitom kod klasične kategorizacije teksta, dajući većinom bolje rezultate od Naivnog Bayesa. Za razliku od Naivnog Bayesa i maksimalne entropije koji koriste vjerojatnost, ova metoda traži hiper-ravninu koja ima najveću marginu razdvajanja klasa -- u našem slučaju pozitivne i negativne klase. Neka je $c_j \in \{1, -1\}$ reprezentacija odgovarajuće klase kojoj propada dokument $d_j$. Tada vektor koji određuje traženu hiper-ravninu možemo zapisati kao:

\[
	\vec{w} = \sum_j \alpha_j c_j \vec{d_j}, \quad \alpha_j \ge 0,
\]

gdje su $\alpha_j$ rješenja dualnog problema optimizacije. Vektori $\vec{d_j}$ za koje je $\alpha_j > 0$ zovu se potporni vektori. Klasifikacija testnih dokumenata vrši se na temelju toga s koje strane hiper-ravnine se oni nalaze.

\section{Očekivani rezultati predloženog projekta}

Želimo naučiti klasifikator da dobro prepoznaje kojoj klasi dokument pripada (provjera pomoću cross validacije). Možda bi mogli napraviti evaluaciju modela -- accuracy, precision/recall, $F_1$ score.

Usporedba -- Naivni Bayes, Maksimalna entropija, SVM.

\newpage

\nocite{*}

\bibliographystyle{unsrt}
\bibliography{sentiment_analysis}

\end{document}
