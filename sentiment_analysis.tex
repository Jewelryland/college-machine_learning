\documentclass[10pt,a4paper,titlepage]{article}

% Napraviti da se normalno može koristiti hrvatski jezik
\usepackage[croatian]{babel}
\usepackage[utf8]{inputenc}
\usepackage[T1]{fontenc}


\usepackage{hyperref} % Linkanje referenci (npr. bibliografija)
\usepackage{float}    % Omogućiti inline slike u tekstu
\usepackage[nottoc,numbib]{tocbibind} % Numeriranje bibliografije

\usepackage{color}    % Omogućiti boje
\usepackage{multicol} % Omogućiti više kolumna teksta (npr. 2)

\usepackage{amsmath}
\usepackage{amsfonts}
\usepackage{amssymb}
\usepackage{graphicx}
\usepackage{parskip}

\title{Analiza stava\\(Projektni prijedlog)}
\author{Lana Arzon\\Janko Marohnić\\Antonio Kovačić}
\date{Zagreb, 2014.}

\begin{document}

\maketitle

\tableofcontents

\newpage

\section{Uvodni opis problema}

Što je analiza stava (semtiment analysis)? - općenito o temi

\section{Cilj i hipoteze istraživanja problema}

Naš zadatak: otkriti da li je stav nekog teksta pozitivan ili negativan.

Konkretno, da li je filmska kritika pozitivna ili negativna.

Koristit ćemo sljedeći skup podataka: Polarity data \\
\url{https://www.cs.cornell.edu/people/pabo/movie-review-data/}

Baseline algorithm:

\begin{itemize}
	\item Tokenization
	\item Feature extraction
	\item Classification using different classifiers
	\begin{itemize}
		\item Naive Bayes
		\item SVM
		\item MaxEnt
	\end{itemize}
\end{itemize}

\section{Pregled dosadašnjih istraživanja}

Možda da spomenemo neke projekte i članke iz ovoga što mi radimo i srodnih područja (subjektivnost, klasifikacija teksta općenito)

\section{Materijali, metodologija i plan istraživanja}

Koristit ćemo već spomenuti skup podataka polarity data koji ćemo možda trebati prilagoditi. Ovaj skup se sastoji od dokumenata(kritike filmova) koji su podijeljeni na dvije klase - pozitivne i negativne kritike. Pomoću njega želimo naučiti naš klasifikator da za dani nepoznati dokument odredi da li je riječ o pozitivnoj ili negativnoj kritici - nadzirano učenje.

Kao značajke ćemo koristiti riječi. Služit ćemo se jednostavnom metodom "Bag of words" koja se temelji samo na broju pojavljivanja značajke, ne uzimajući u obzir poredak i ostale značajke. Dakle, pretpostavljamo da su značajke nezavisne.

Realizacija projekta - u programskom jeziku Python korištenjem alata NLTK (Natural Language Toolkit).

\subsection{Naive Bayes}

Bayesovo pravilo primijenjno na dokument $d$ i klasu $c$:

\[P(c|d) = \frac{P(d|c)P(c)}{P(d)}.\]

Klasa kojoj najvjerojatnije pripada dokument (maximum a posteriori class):

\[c_{\text{MAP}} = \underset{{c \in C}}{\text{argmax }} \frac{P(d|c)P(c)}{P(d)}
= \underset{{c \in C}}{\text{argmax }} P(d|c)P(c).\] 

Odnosno, ako dokument $d$ reprezentiramo pomoću značajki $x_1, x_2, \ldots, x_n$:

\[c_{\text{MAP}} = \underset{{c \in C}}{\text{argmax }} P(x_1, x_2, \ldots, x_n|c)P(c).\] 

\subsection{MaxEnt (možda)}

Ovo nisam još proučila

\subsection{SVM (možda)}

Ni ovo :)

\section{Očekivani rezultati predloženog projekta}

Želimo naučiti klasifikator da dobro prepoznaje kojoj klasi dokument pripada. Možda bi mogli napraviti evaluaciju modela - accuracy, precision/recall, $F_1$ score.

Usporedba - Naive Bayes, MaxEnt, SVM.

\newpage

\nocite{Pang:2002:TUS:1118693.1118704}
\nocite{bird2009natural}

\bibliographystyle{unsrt}
\bibliography{sentiment_analysis}

\end{document}
